\chapter{Revisão de Teoria Quâtica de Campos}

Neste capítulo apresentamos uma revisão sistemática do efeito Casimir à luz da teoria quântica de campos. O desenvolvimento inicia-se com a formulação lagrangiana de campos clássicos, a discussão de simetrias contínuas e suas consequências via o teorema de Nöther. Em seguida, procede-se à quantização de campos escalares e à análise da energia do vácuo na presença de fronteiras. A partir desse formalismo, o efeito Casimir é estudado explicitamente no caso de placas paralelas, conduzindo à obtenção da energia do vácuo e da pressão associada.

\section{Teorema de Nöther}

\par O teorema de Nöther estabelece a relação fundamental entre simetrias contínuas da ação e leis de conservação, desempenhando um papel central na teoria clássica e quântica de campos. Em particular, as simetrias de translação no espaço-tempo conduzem à conservação da energia e do momento, que são descritas de forma unificada pelo tensor energia-momento. Nesta seção, revisamos o teorema de Nöther no formalismo lagrangiano e derivamos a expressão do tensor energia-momento para campos clássicos, com o objetivo de fundamentar sua utilização posterior no cálculo da energia de Casimir em sistemas com condições de contorno.

Seja um sistema dinâmico descrito pela ação

\begin{equation}
 S = \int d^4 x \mathcal{L} \label{eq:acao}
\end{equation}

onde assumimos, 

\begin{equation}
\mathcal{L} = \mathcal{L}(\phi (x), \partial_\mu \phi (x)) \label{eq:lagphi}
\end{equation}

Sob uma tranformação geral da forma
\begin{equation}
\begin{aligned}
&x^\mu \rightarrow x'^\mu \\
&\phi (x) \rightarrow \phi'(x') \\
&\partial_\mu \phi (x) \rightarrow \partial'_\mu \phi' (x') \\ \label{eq:tranformacoes}
\end{aligned}
\end{equation}

dizemos que a dinâmica descrita pela ação \eqref{eq:acao} é invariante sob transformações \eqref{eq:tranformacoes} se a ação não muda sob essa tranformações. A saber, se

\begin{equation}
S = \int d^4 x \mathcal{L} (\phi (x), \partial_\mu \phi (x)) = \int d^4 x' \mathcal{L} (\phi' (x'), \partial'_\mu \phi' (x'))
\end{equation}

então, as transformações \eqref{eq:tranformacoes} definem uma simetria do sistema. É claro que neste caso, as equações de Euler-Lagrange para os sietmas com e sem linha continuariam invariantes. Vale a pena enfatizar aqui que \ref{eq:tranformacoes} inclui uma classe muito interessante de transformações onde as coordenadas espaço-tempo não mudam, ou seja, 

\begin{equation}
x'^\mu = x^\mu , \label{eq:Xinvariante}
\end{equation}

e somente as variáveis dinâmicas da teoria se transformam. Tais transformações são conhecidas como transforamções de simetria interna para serem contrastadas com as transformações espaço-tempo onde as coordenadas espaço-tempo se transformam juntamente com comm as variáveis dinâmicas, como em \eqref{eq:tranformacoes}. Nossa discussão de simetrias se aplica tanto às transformações espaço-tempo, onde as coordenadas espaço-tempo são transformadas, como as transformações de simetria interna, onde as coordenadas espaço-tempo não são afetadas pelas transformações.

Simetrias têm consequências interessantes para transformações contínuas. Então, Consideremos uma transformação infinitesimal com

\begin{equation}
\left| \frac{\partial x'}{\partial x} \right| = 1, \label{eq:deXdevX}
\end{equation}

o que vale para a maioria das simetrias globais de espaço-tempo , bem como para as simetrias internas. Nesse caso, invariância da ação sob as formas infinitesimais das transformações em \eqref{eq:tranformacoes} implicaria

\begin{equation}
\begin{aligned}
&\delta S = &\int d^4 x' \mathcal{L} (\phi' (x'), \partial'_\mu \phi' (x')) - \int d^4 x \mathcal{L} (\phi (x), \partial_\mu \phi (x)) = 0, \\
&ou, &\int d^4 x \mathcal{L} (\phi' (x), \partial_\mu \phi' (x)) - \int d^4 x \mathcal{L} (\phi (x), \partial_\mu \phi (x)) = 0, \\
&ou, &\int d^4 x ( \mathcal{L} (\phi' (x), \partial_\mu \phi' (x)) - \mathcal{L} (\phi (x), \partial_\mu \phi (x)) ) = 0, \\ \label{eq:deltaS}
\end{aligned}
\end{equation}

Claramente, \eqref{eq:deltaS} será válida se 

\begin{equation}
\mathcal{L} (\phi'(x), \partial_\mu \phi'(x)) - \mathcal{L} (\phi(x), \partial_\mu \phi(x)) = \partial_\mu K^\mu, \label{eq:deltaL1}
\end{equation}

que deve valer independentemente do uso de equações de movimento para o sistema que estamos vendo. Isto é bem geral e é possível ter transformações de simetria pelo qual $K^\mu$.
Por outro lado, definir a transformação infinitesimal no campo como

\begin{equation}
\phi'(x) - \phi(x) = \delta \phi (x) \label{eq:transformacaocampo}
\end{equation}

de modo que

\begin{equation}
\delta (\partial_\mu \phi(x)) = \partial_\mu \phi' (x) - \partial_\mu \phi(x)  = \partial_\mu \delta \phi(x)
\end{equation}

podemos calcular explicitamente, uma vez que mantemos somente os termos lineares, porque $\delta \phi(x)$ é infinitesimal,

\begin{equation}
\begin{aligned}
&\mathcal{L} (\phi'(x), \partial_\mu \phi'(x)) - \mathcal{L} (\phi(x), \partial_\mu \phi(x)) \\
&= \mathcal{L} (\phi(x), \partial_\mu \phi(x)) + \delta \phi(x) \frac{\partial \mathcal{L}}{\partial \phi(x)} + \delta (\partial_\mu \phi (x)) \frac{\partial \mathcal{L}}{\partial \partial_\mu \phi(x)} - \mathcal{L} (\phi(x), \partial_\mu \phi(x)) \\
&= \delta \phi(x) \frac{\partial \mathcal{L}}{\partial \phi(x)} + \delta (\partial_\mu \phi (x)) \frac{\partial \mathcal{L}}{\partial \partial_\mu \phi(x)} \\
&= \delta \phi(x) \partial_\mu \frac{\partial \mathcal{L}}{\partial \partial_\mu  \phi(x)} + \delta (\partial_\mu \phi (x)) \frac{\partial \mathcal{L}}{\partial \partial_\mu \phi(x)} \\
&= \partial_\mu \left( \delta \phi (x) \frac{\partial \mathcal{L}}{\partial \partial_\mu \phi (x)} \right) . \\ \label{eq:deltaL2}
\end{aligned}
\end{equation}

Usamos a equação de Euler-Lagrange na penúltima linha. Comparando \eqref{eq:deltaL1} e \eqref{eq:deltaL2}, obtemos 

\begin{equation}
\begin{aligned}
&\partial_\mu \left( \delta \phi (x) \frac{\partial \mathcal{L}}{\partial \partial_\mu \phi (x)} \right) = \partial_\mu K^\mu \\
ou, \quad &\partial_\mu \left( \delta \phi (x) \frac{\partial \mathcal{L}}{\partial \partial_\mu \phi (x)} -  K^\mu \right) = 0 \\
\end{aligned}
\end{equation}

Isso mostra que sempre que houver uma simetria contínua associada com o sistema, podemos definir uma corrente

\begin{equation}
\partial J^\mu =  \delta \phi (x) \frac{\partial \mathcal{L}}{\partial \partial_\mu \phi (x)} -  K^\mu
\end{equation}

que é conservado, ou seja,

\begin{equation}
\partial_\mu J^\mu = 0
\end{equation}

\subsection{Translação espaço-tempo}

Como um exemplo das consequências do teorema de Nöther, consideremos o caso simples de uma translação infinitesima global espaço-tempo definido por

\begin{equation}
\begin{aligned}
&x^\mu \rightarrow x'^\mu = x^\mu + \epsilon^\mu \\
ou, \quad &\delta x^\mu = x'^\mu -x^\mu = \epsilon ^\mu , \label{eq:spacetimetranslation}
\end{aligned}
\end{equation}

Como estamos lidando com um campo escalar, 

\begin{equation}
\phi'(x') = \phi(x)
\end{equation}

E nesse caso nós obtemos a mudança no campo para corresponder a

\begin{equation}
\begin{aligned}
\delta \phi(x) &= \phi'(x) - \phi(x) = \phi'(x) - \phi'(x') \\
&= -(\phi'(x') - \phi'(x)) = - \epsilon^\mu \partial_\mu \phi'(x) \\ 
&= - \epsilon^\mu \partial_\mu \phi(x) \\ \label{eq:deltaPhi=epsilonPartialPhi}
\end{aligned}
\end{equation} 
Fizemos $\phi'(x)=\phi(x)$, porque multiplicar o parâmetro infinitesimal $\epsilon^\mu$ no lado direito da equação nos permite ignorar termos de mais altas ordens. Com isso, podemos calcular explicitamente a mudança na densidade lagrangiana
\begin{equation}
\begin{aligned}
\mathcal{L} (\phi'(x), \partial_\mu \phi'(x)) &= \mathcal{L} (\phi(x), \partial_\mu \phi(x)) \\
&= \delta \phi(x) \frac{\partial \mathcal{L}}{\partial \phi(x)} + (\partial_\nu \delta \phi(x)) \frac{\partial \mathcal{L}}{\partial \partial_\nu \phi(x)} \\
&= -\epsilon^\mu \partial_ \phi(x) \frac{\partial \mathcal{L}}{\partial \phi(x)} -\epsilon^\mu (\partial_\mu \partial_\nu \phi(x)) \frac{\partial \mathcal{L}}{\partial \partial_\nu \phi(x)} \\
&= -\epsilon^\mu \partial_\mu \mathcal{L} = \partial_\mu K^\mu , \\ \label{eq:partialLPartialK}
\end{aligned}
\end{equation}
Vemos que uma vez que a mudança na densidade lagrangiana é uma divergência total, a ação é invariante sob translações infinitesimais que definem uma simetria do sistema. Podemos identificar agora de \eqref{eq:partialLPartialK} que 

\begin{equation}
K^\mu = -\epsilon^\mu \mathcal{L} 
\end{equation}
Perceba que sob a transformação \eqref{eq:deltaPhi=epsilonPartialPhi} ,

\begin{equation}
\delta \phi (x) \frac{\partial \mathcal{L}}{\partial \partial_\mu \phi (x)  } = -\epsilon^\nu (\partial_\nu \phi(x) \frac{\partial \mathcal{L}}{\partial \partial_\mu \phi (x)  }
\end{equation}

de maneira que obtemos a corrente associada com a transormação de simetria, dependendo do parâmetro de transformação,

\begin{equation}
\begin{aligned}
J^\mu_\epsilon &= \delta \phi (x) \frac{\partial \mathcal{L}}{\partial \partial_\mu \phi (x)  } - K^\mu \\
&= -\epsilon^\nu (\partial_\nu \phi(x) \frac{\partial \mathcal{L}}{\partial \partial_\mu \phi (x)  } + \epsilon^\mu \mathcal{L} \\
&= -\epsilon_\nu \left( (\partial_\nu \phi(x) \frac{\partial \mathcal{L}}{\partial \partial_\mu \phi (x)  } - \eta^{\mu\nu}  \mathcal{L} \right) = -\epsilon_\nu T^{\mu\nu} \\
\end{aligned}
\end{equation}

Fica claro, portanto, que a corrente conservada independente do parâmetro pode ser identificada com

\begin{equation}
T^{\mu\nu} = (\partial_\nu \phi(x) \frac{\partial \mathcal{L}}{\partial \partial_\mu \phi (x)  } - \eta^{\mu\nu}  \mathcal{L} \label{eq:stressTensor}
\end{equation}

Que é o tensor energia momemto. Pode-se verificar que 

\begin{equation}
\partial_\mu T^{\mu\nu} = 0
\end{equation}

Além disso, verifica-se que ele é simétrico, ou seja,

\begin{equation}
T^{\mu\nu} = T^{\nu\mu}
\end{equation}

O operador energia-momento pode ser obtido a partir de 

\begin{equation}
P^\mu = \int d^3x T^{0\mu}
\end{equation}

Ou seja, o tensor energia-momento e o operador energia-momento são, respectivamente, a corrente e a carga associadas à simetria de transnalação de um sistema.

Para a teoria de Klein-Gordon livre de interações, lembremos que a densidade Lagrangiana tem a forma, 

\begin{equation}
\mathcal{L} = \frac{1}{2} \partial_\mu \phi \partial^\mu \phi - \frac{m^2}{2} \phi^2 . 
\end{equation}
Então obtemos, 

\begin{equation}
\frac{\partial \mathcal{L}}{\partial \partial_\mu \phi(x)} = \partial^\mu \phi(x) , 
\end{equation}

E isso leva à forma explícita do tensor de tensão, a partir de \eqref{eq:stressTensor} 

\begin{equation}
T^{\mu\nu} = \partial^\nu \phi(x) \partial^\mu \phi(x) - \eta^{\mu\nu} \mathcal{L} = T^{\nu\mu}
\end{equation}

Notamos que, para a teoria de um sistema livre de interações,

\begin{equation}
\begin{aligned}
P^0 &= \int d^3x T^{00} \\
&= \int d^3x \left[ \left( \dot{\phi}(x) \right)^2 - \frac{1}{2} \left( \dot{\phi}(x) \right)^2 + \frac{1}{2} \nabla \phi \cdot \nabla \phi + \frac{1}{2} \phi^2 \right] \\
&= \int d^3x \left[ \frac{1}{2} \left( \dot{\phi}(x) \right)^2 + \frac{1}{2} \nabla \phi \cdot \nabla \phi + \frac{1}{2} \phi^2 \right] \\
&= H \\
P^i &= \int d^3x T^{0i} \\
&= \int d^3x \partial^i \phi(x) \dot{\phi(x)}, \\
\boldsymbol{P} &= - \int d^3x \nabla \phi(x) \dot{\phi}(x) \label{eq:HeP}
\end{aligned}
\end{equation}
\section{O efeito Casimir escalar para placas paralelas}

Nesta seção da revisão iremos desenvolver o formalismo do campo escalar necessário para o estudo do efeito Casimir entre placas paralelas, desde a equação de Klein--Gordon e as condições de contorno até a obtenção da energia do vácuo e da pressão de Casimir.

%-----------------------------------------------------------------------------------
%-----------------------------------------------------------------------------------
\subsection{Equação do campo}
%-----------------------------------------------------------------------------------
%-----------------------------------------------------------------------------------

A equação de Klein-Gordon para um campo escalar real $\phi(x)$ é dada por

\begin{equation}
(\Box + m^2)\phi(x)= 0, \label{eq:KLeinGordonEq}
\end{equation}

onde o operador D'Alembertiano quadridimensional é definido como

\begin{equation}
\Box \equiv \eta^{\mu\nu} \frac{\partial^2}{\partial x^\mu \partial x^\nu} = \partial^\nu \partial_\nu 
\end{equation}

e m é a massa do campo. Se houver alguma fonte externa, a equação \eqref{eq:KLeinGordonEq} é generalizada para 

\begin{equation}
(\Box + m^2)\phi(x)= \Upsilon (x) \label{eq:GenKleinGordonEq}
\end{equation}

Ambas as equações \eqref{eq:KLeinGordonEq} e \eqref{eq:GenKleinGordonEq} são obtidas das equações de Euler-Lagrange da ação de um campo escalar 

\begin{equation}
S[\phi] = \int d^4x \mathcal{L} (x) = \int d^4x \left( \frac{1}{2} \partial^\nu \phi \partial_\nu - \frac{m^2}{2} \phi^2  + \Upsilon \phi \right)
\end{equation}
Usando integração por partes, a ação pode ficar na forma

\begin{equation}
S [\phi] = \int d^4x \left( -\frac{1}{2}\phi K \phi + \Upsilon \phi \right) 
\end{equation}

onde o operador
\begin{equation}
K \equiv K(x) = \Box + m^2
\end{equation}

é o kernel da ação.

%-----------------------------------------------------------------------------------
%-----------------------------------------------------------------------------------
\subsection{Quantização canônica do campo e a energia de vácuo como uma expansão em modos}
%-----------------------------------------------------------------------------------
%-----------------------------------------------------------------------------------

Para contornos estáticos suaves de qualquer forma geométrica, sempre é possível introduzir soluções de frequências negativas e positivas da equação de Klein-Gordon

\begin{equation}
\phi^{(+)}_J (t, \boldsymbol{r}) = \frac{1}{\sqrt{2\omega_J}} e^{-i\omega_J t} \boldsymbol{\Phi}_J (\boldsymbol{r}) , \quad \phi^{(-)}_J (t, \boldsymbol{r}) = \left[ \phi^{(+)}_J (t, \boldsymbol{r}) \right]^* \label{eq:campos}
\end{equation}

Onde J é um índice coletivo para o vetor de onda generalizado, e $\boldsymbol{\Phi} (\boldsymbol{r})$ satifaz algumas condições de contorno de uma superfície S. Se ela for aplicada à \eqref{eq:KLeinGordonEq}, ficamos com 

\begin{equation}
- \nabla^2 \boldsymbol{\Phi} (\boldsymbol{r}) = \Lambda_J \boldsymbol{\Phi}_J (\boldsymbol{r}), \quad \Lambda_J \equiv \omega_J - m^2. \label{eq:spatialEq}
\end{equation}

As funções \eqref{eq:campos} satifazem as condições de normalização

\begin{equation}
\left( \phi^{(\pm)}_{J}(x), \phi^{(\pm)}_{J'}(x) \right) = \pm \delta_{JJ'}, \quad \left( \phi^{(\pm)}_{J}(x), \phi^{(\mp)}_{J'}(x) \right) = 0,
\end{equation}

onde 

\begin{equation}
(f,g) = i \int_V d^3x \left( f^*\frac{\partial g}{\partial x^0} - \frac{\partial f^*}{\partial x^0}g \right) \label{eq:scalarProduct}
\end{equation}

é definido como o produto escalar de duas funções. De \eqref{eq:scalarProduct} também obtemos a condição de normalização para as soluções $\boldsymbol{\Phi}(x)$ do problema de contorno,

\begin{equation}
\int_V d^3x \boldsymbol{\Phi}^*_J(\boldsymbol{r}) \boldsymbol{\Phi}_{J'}(\boldsymbol{r}) = \delta_{JJ'} \label{eq:normalization}
\end{equation}

Seguindo o processo de quantização canônica, segue-se o operador campo como a soma dos modos

\begin{equation}
\phi(x) = \sum_J \left[ \phi_J^{(+)}(x)a_J + \phi_J^{(-)}(x)a^+_J \right] \label{eq:campogeral}
\end{equation}

onde $a_J$ e $a^+_J$ são, respectivamente, os operadores de criação e de aniquilação de uma partícula com números quânticos indicados pelo índice coletivo J. O somatório sobre J pode se tornar uma integral se alguns (ou todos) os números quânticos forem contínuos. Eles satifazem as relações de comutação

\begin{equation}
[a_J, a^+_{J'}] = \delta_{JJ'}, \quad [a_J, a_{J'}] = 0. \label{eq:commutrel}
\end{equation}

O estado de vácuo do campo é definido por

\begin{equation}
a_J \ket{0} = 0 \label{eq:aKet0=0}
\end{equation}

E para obter estados com partículas usamos o operador de criação no vácuo. Por exemplo,

\begin{equation}
\ket{1} = a^+_J \ket{0} \label{eq:1=a+ket0}
\end{equation}

No caso em que não há nenhuma condição de contorno, o índice J coincide como o vetor de onda, $J \equiv \boldsymbol{k} = (k^1, k^2, k^3)$, as frequências de oscilação são dadas por $\omega_J = \omega_k = (m^2 + \boldsymbol{k}^2)^{\frac{1}{2}}$, e

\begin{equation}
\boldsymbol{\Phi}_J (\boldsymbol{r}) = \boldsymbol{\Phi}_{\boldsymbol{k}} (\boldsymbol{r}) = 
\frac{e^{-i \boldsymbol{k} \cdot \boldsymbol{r}}}{(2\pi)^{\frac{3}{2}}} \label{eq:phiLivre}
\end{equation}

Neste caso, o símbolo $\delta_{JJ'}$ nas equações anteriores em que ele aparece deve ser mudado para $\delta^3(\boldsymbol{k} - \boldsymbol{k'})$. O estado de vácuo do campo escalar no espaço de Minkowski livre de interações é definido por 

\begin{equation}
a_{\bm{k}} \ket{0_M} = 0.
\end{equation}

A energia de densidade no vácuo do campo escalar na presença de contornos é o valor esperado da componente 00 do tensor energia momento \eqref{eq:stressTensor} no estado de vácuo,

\begin{equation}
\bra{0} T_{00}^{(0)}(x) \ket{0} = \frac{1}{2} \left\langle 0 \left| \left[ \sum_{\mu=0}^3 \left( \frac{\partial \phi}{\partial x^\mu } \right)^2 + m^2\phi^2 \right] \right| 0 \right\rangle \label{eq:tensorMeanvalue}
\end{equation}

Substituindo a equação \eqref{eq:campogeral} na \eqref{eq:tensorMeanvalue} e usando as equações \eqref{eq:campos}, \eqref{eq:commutrel} e \eqref{eq:aKet0=0}, obtemos

\begin{equation}
\bra{0} T_{00}^{(0)}(x) \ket{0} = \sum_J \frac{1}{4\omega_J} \left[ (\omega_J^2 + m^2)\bm{\Phi}_J (\bm{r}) \bm{\Phi}_J^*(\bm{r}) + \sum_{k=1}^3 \frac{\partial \bm{\Phi}_J(\bm{r})}{\partial x^k} \frac{\partial \bm{\Phi}^*_J(\bm{r})}{\partial x^k}     \right] \label{eq:<T>}
\end{equation}

Consideremos agora a energia total do vácuo do campo escalar no volume de quantização V. Assumimos que as funções $\bm{\Phi}_J(\bm{r})$ a condição de contorno de Dirichlet ou de Neumann numa superfície S. Ao integrar a equação \eqref{eq:<T>} sobre V usando \eqref{eq:spatialEq} e \eqref{eq:normalization}, obtemos

\begin{equation}
E_0 = \int_V d^3x \bra{0} T_{00}^{(0)} \ket{0} = \frac{1}{2} \sum_J \omega_J \label{eq:modeSum}
\end{equation}

Para o espaço de Minkowski livre de interação sem fronteiras, a densidade de energia de vácuo é obtida das equações \eqref{eq:phiLivre} e \eqref{eq:<T>} 

\begin{equation}
\bra{0_M} T_{00}^{(0)} \ket{0_M} = \frac{1}{2} \int \frac{d^3x}{(2\pi)^3} \omega_k . \label{eq:freeSol}
\end{equation}

A energia total do vácuo no volume V é

\begin{equation}
E_0 = \int_V d^3x \bra{0_M} T_{00}^{(0)} \ket{0_M} = \frac{1}{2} \int \frac{d^3x}{(2\pi)^3} \omega_k V.
\end{equation}

%-----------------------------------------------------------------------------------
%-----------------------------------------------------------------------------------
\subsection{Aplicação em placas paralelas}
%-----------------------------------------------------------------------------------
%-----------------------------------------------------------------------------------


Consideremos a energia de vácuo de Casimir de um campo escalar numa configuração de dois planos paralelos em três dimensões com Dirichlet.

Começamos das quantidades locais e damos especial atenção às regiões do espaço externo à configuração entre as placas. Sejam duas placas em z=0 e z=a.
Na região entre as placas, o conjunto completo ortonormal de soluções para a equação \eqref{eq:KLeinGordonEq} satisfazendo a condição de contorno de Dirichlet $\bm{\Phi}_J(x,y,0)=\bm{\Phi}_J(x,y,a) = 0$ é dada pela equação \eqref{eq:campos}, onde

\begin{equation}
\begin{aligned}
\bm{\Phi}_J(\bm{r})=\bm{\Phi}_{k_\perp, n}(\bm{r}) = \frac{1}{\pi \sqrt{2a}}e^{i(k_xx + k_yy)} sen(k_{zn}z), \\
\omega^2_{k_\perp, n} = m^2 + k_{\perp}^2 + k_{zn}^2, \quad k_{\perp}^2 = k_x^2 + k_y^2, \quad k_{zn} = \frac{\pi n}{a}.
\end{aligned}
\end{equation}
Onde a discretização de k na direção de z é devido às condições de contorno impostas durante a solução da equação \eqref{eq:KLeinGordonEq}. Substituindo essas soluções na equação \eqref{eq:<T>} obtemos a densidade de energia no vácuo 

\begin{equation}
\bra{0} T_{00}^{(0)} \ket{0} = \frac{1}{2a} \int_0^\infty \frac{k_\perp dk_\perp}{2\pi} \sum_{n=1}^\infty  \left[\omega_{k_\perp, n} - \frac{m^2 + k_\perp^2}{\omega_{k_\perp, n}} cos(2k_{zn}z) \right] \label{eq:comPlacas}
\end{equation}
Onde mudamos para o sistema de coordenadas polares e usamos a relação $sen^2(x) = \frac{1-cos(2x)}{2}$

O último termo é oscilatório e não contribui para a energia total. No espaço de Minkowski livre de interações, como vimos, o conjunto completo ortonormel de soluções é dado pela equação \eqref{eq:freeSol}, que mudando para coordenadam em k, nos leva a 
\begin{equation}
\bra{0_M} T_{00}^{(0)} \ket{0_M} = \frac{1}{2} \int \frac{d^3k}{(2\pi)^3} \omega_k . \label{eq:semPlacas}
\end{equation}

onde $d^3k = dk_xdk_ydk_z$, $\omega_k = (m^2+\bm{k}^2)^{1/2}$. A densidade de energia de Casimir

\begin{equation}
\epsilon(z) = \bra{0} T_{00}^{(0)}(z) \ket{0} - \bra{0_M} T_{00}^{(0)} \ket{0_M}
\end{equation}

pode ser encontrada a partir de \eqref{eq:comPlacas} e \eqref{eq:semPlacas} usando a fórmula de Abel-Plana

\begin{equation}
\sum_{n=0}^\infty F(n) - \int_0^\infty F(t)dt = \frac{1}{2} F(0) + i \int_0^\infty \frac{dt}{e^{2\pi t} - 1} [F(it)  - F(-it)], \label{eq:abelPlana}
\end{equation}

de modo que obtemos o resultado

\begin{equation}
\begin{aligned}
\epsilon(z) &= -\frac{1}{2} \int_0^\infty  \frac{k_\perp dk_\perp}{2\pi} \bigg[ \frac{\sqrt{m^2 + k_\perp^2}}{2} + \frac{2\pi}{a} \int_A^\infty \frac{\sqrt{t^2 - A^2}}{e^{2\pi t} - 1}dt \\
 &+ (m^2 + k_\perp^2)\sum_{n=1}^\infty \frac{cos(2k_{zn}z)}{\omega_{k_\perp, n}} \bigg], \label{eq:denEnPlacas}
\end{aligned}
\end{equation}

Onde,

\begin{equation}
A \equiv \frac{a\sqrt{m^2 + k_\perp^2}}{\pi}, \quad t \equiv \frac{ak_z}{\pi}.
\end{equation}

Se descartarmos o termos oscilatório de \eqref{eq:denEnPlacas}, a energia de Casimir por unidade de área das placas é dada por

\begin{equation}
\begin{aligned}
E(a) &= \int_0^a dz \epsilon(z) = -\frac{1}{2} \int_0^\infty \frac{k_\perp dk_\perp}{2\pi} \bigg( \frac{\sqrt{m^2 + k_\perp^2}}{2} \\
&+ \frac{2\pi}{a} \int_A^\infty \frac{\sqrt{t^2 - A^2}}{e^{2\pi t}-1}dt \bigg)
\end{aligned}
\end{equation}

Para um campo sem massa, temos $A = ak_\perp / \pi$, de modo que depois de integramos temos, 

\begin{equation}
E(a) = -\frac{\pi^2}{1440a^3} - \frac{1}{8\pi} \int_0^\infty k_\perp^2 dk_\perp. \label{eq:EnergiaEntrePlacas}
\end{equation}

Uma vez que essa já é a energia por undide de área, para calcular a pressão basta derivar em relação à distância entre as placas

\begin{equation}
P(a) = - \frac{\partial E(a)}{\partial a}
\end{equation}

Portanto a pressão entre as placas é

\begin{equation}
P(a) = - \frac{\pi^2}{480a^4} \label{eq:pressaoEntreAsPlacas}
\end{equation}
%-----------------------------------------------------------------------------------
%-----------------------------------------------------------------------------------
\section{Equação do campo}
%-----------------------------------------------------------------------------------
%-----------------------------------------------------------------------------------

A equação de Klein-Gordon para um campo escalar real $\phi(x)$ é dada por

\begin{equation}
(\Box + m^2)\phi(x)= 0, \label{eq:KLeinGordonEq}
\end{equation}

onde o operador D'Alembertiano quadridimensional é definido como

\begin{equation}
\Box \equiv \eta^{\mu\nu} \frac{\partial^2}{\partial x^\mu \partial x^\nu} = \partial^\nu \partial_\nu 
\end{equation}

e m é a massa do campo. Se houver alguma fonte externa, a equação \eqref{eq:KLeinGordonEq} é generalizada para 

\begin{equation}
(\Box + m^2)\phi(x)= \Upsilon (x) \label{eq:GenKleinGordonEq}
\end{equation}

Ambas as equações \eqref{eq:KLeinGordonEq} e \eqref{eq:GenKleinGordonEq} são obtidas das equações de Euler-Lagrange da ação de um campo escalar 

\begin{equation}
S[\phi] = \int d^4x \mathcal{L} (x) = \int d^4x \left( \frac{1}{2} \partial^\nu \phi \partial_\nu - \frac{m^2}{2} \phi^2  + \Upsilon \phi \right)
\end{equation}
Usando integração por partes, a ação pode ficar na forma

\begin{equation}
S [\phi] = \int d^4x \left( -\frac{1}{2}\phi K \phi + \Upsilon \phi \right) 
\end{equation}

onde o operador
\begin{equation}
K \equiv K(x) = \Box + m^2
\end{equation}

é o kernel da ação.
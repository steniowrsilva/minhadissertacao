\section{Acoplamento Éter-escalar}
O campo éter é um campo vetorial que pode ser acoplado a um campo escalar, modificando a dinâmica deste último. Ele é um vetor do tipo espaço de cinco dimensões $u^a = (0,0,0,0,v)$, onde a quinta dimensão, $x_5$, é compactificada num círculo de raio $b$. O vetor tem dimensão de massa e é constante, ou seja, ele dá um direção preferencial, consequentemente, quebrando a simetria de Lorentz. Deste modo, a lagrangiana do sistema, associada à interação entre o campo escalar e o éter é dada por
\begin{equation}
  \mathcal{L} = -\frac{1}{2}(\partial_a\phi)^2 - \frac{1}{2\mu^2_\phi} u^a u^b \partial_a \phi \partial_b \phi \label{eq:LagrangeAether}
\end{equation}

onde $\mu_\phi$ é um parâmetro de acoplamento com dimensão de massa que controla a intensidade do acoplamento entre o campo escalar e o éter. Substituindo \eqref{eq:LagrangeAether} nas equações de Euler-Lagrange,
\begin{equation}
    \partial_a \left( \frac{\partial \mathcal{L}}{\partial (\partial_a \phi)} \right) - \frac{\partial \mathcal{L}}{\partial \phi} = 0,
\end{equation}

Obtemos a equação de movimento para o campo escalar acoplado ao éter,
\begin{equation}
    \begin{aligned}
            \partial_a \partial^a \phi + \frac{1}{\mu^2_\phi} u^a u^b \partial_a \partial_b \phi = 0, \\
            \partial_a \partial^a \phi = -\mu_\phi^{-2} \partial_a (u^a u^b \partial_b \phi)     \end{aligned}
\end{equation}

Ao usar a solução de onda plana $\phi \sim e^{i k_a x^a} = e^{ik_\mu k^\mu + ik_5k^5}$, e $u_a u^a = v^2$, obtemos a relação de dispersão modificada,
\begin{equation}
    -k_\mu k^\mu = (1 + \alpha_\phi^2) k_5^2
\end{equation}

Onde $\alpha_\phi = \frac{v}{\mu_\phi}$ é um parâmetro adimensional. Essa relação de dispersão modificada implica que a presença do campo éter altera a energia dos modos do campo escalar, o que, por sua vez, afeta a energia de vácuo e a força de Casimir entre as placas paralelas. 

Se impusermos a condição de contorno quasi-periódica 
\begin{equation}
    \phi(t,x,y,z,x_5) = e^{-2\pi \beta i} \phi(t,x,y,z,x_5 + b), \label{eq:quasiperiodic}
\end{equation}
e a condição de contorno de Neumann
\begin{equation}
    \partial_z \phi (t,x,y,z, x^5)_{z=0} = 0, \quad \partial_z \phi (t,x,y,z, x^5)_{z=a} = 0,
\end{equation}
na solução de onda plana do campo escalar os momentos na direção de $x^5$ e $z$ são discretizados. $\beta$ é uma fase que pode variar entre $0$ e $1$, permitindo interpolar entre condições de contorno periódicas ($\beta=0$) e antiperiódicas ($\beta=1/2$). Ela generaliza as condições de contorno periódica e antiperiódica já bem conhecidas $\beta=0$ e $\beta=1/2$, respectivamente. Então, pela condição quasi-periódica, a dimensão extra é compactificada num círculo $\mathbb{S}^1$.
A presenção das condições de contorno de Neumann na direçao z permite a medição do efeito no laboratório. Desse modo, os efeitos da dimensão extra e do campo éter podem ser detectados como modificações na força de Casimir entre as placas paralelas. 

Estamos lidando com o espaço-tempo $\mathbb{R}^4 \times \mathbb{S}^1$, onde $\mathbb{R}^4$ é o espaço-tempo de Minkowski e $\mathbb{S}^1$ é a dimensão extra compactificada. A condição \eqref{eq:quasiperiodic} aplicada ao fator $e^{i k_5 x^5}$ da solução $\phi \sim e^{i k_\mu x^\mu + i k_5 x^5}$, da

\begin{equation}
    e^{ik_5 x^5} = e^{-2\pi i \beta} e^{k_5 (x^5 + b)} \implies 1= e^{-2\pi i \beta} e^{i k_5 b} \implies e^{i k_5 b} = e^{2\pi i \beta} 
\end{equation}
portanto o espectro é

\begin{equation}
    k_n = \frac{2\pi}{b} (n + \beta), \quad n \in \mathbb{Z}
\end{equation}
onde fizemos $k_5 \equiv k_n$. Já para a dependência de $\phi$ em relação a z em modo estacionário devido à condição de contorno de Neumann , temos, por sepração de variáveis, 
\begin{equation}
    \phi \sim e^{-i\omega t}e^{ik_x x + ik_y y} Z(z) e^k_x x^5
\end{equation}

A parte em z satisfaz 
\begin{equation}
     Z''(z) + k_z^2 Z(z) = 0 \label{eq:Z}, \quad k_z^2 = \omega^2 - k_x^2 - k_y^2 - (1 + \alpha_\phi^2)k_n^2
\end{equation}

Usando as condiçoes de Neumann 
\begin{equation}
   \left.\frac{dZ}{dz} \right|_{z=0} = 0, \quad \left.\frac{dZ}{dz} \right|_{z=a} = 0, 
\end{equation}
em \eqref{eq:Z}, obtemos que 
\begin{equation}
    k_z = k_m =  \frac{m\pi}{a}, \quad m \in \mathbb{Z}
\end{equation}

Então, a relação de dispersão submetida a ambas as condições de contorno é dada por
\begin{equation}
    -k^\mu k_\mu = k_m^2 + (1 +\alpha^2_\phi)k_n^2, \label{eq:kmukmu}
\end{equation}
Consequentemente, as autofrequências do sistema são
\begin{equation}
    \omega_{n} ^2 = k_x^2 + k_y^2 + k_m^2 + (1 +\alpha^2_\phi)k_n^2 \label{eq:omega}
\end{equation}
Usaremos essa expressão para calcular a energia de Casimir na próxima seção.


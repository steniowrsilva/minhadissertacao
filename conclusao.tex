\chapter{Conclusões e Perpectivas}
Nesta dissertação apresentamos uma análise sistemática do efeito Casimir no contexto da teoria quântica de campos, com foco em campos escalares submetidos a condições de contorno. A abordagem adotada baseou-se em uma construção conceitual progressiva, iniciando pela formulação lagrangiana de campos clássicos e pela análise das simetrias contínuas da ação, culminando na quantização do campo e na interpretação da energia do vácuo como uma soma sobre modos normais.

A revisão do teorema de Noether mostrou-se essencial para fundamentar a definição do tensor energia-momento e, consequentemente, da energia associada ao sistema. Esse formalismo forneceu a base necessária para a quantização canônica do campo escalar e para a identificação rigorosa da energia de ponto zero, cuja dependência em relação às condições de contorno dá origem ao efeito Casimir. O caso específico de placas paralelas permitiu ilustrar de forma clara como a modificação do espectro de modos do campo conduz a uma energia de vácuo finita e a uma pressão mensurável entre as placas.

Os resultados apresentados confirmam que o efeito Casimir pode ser compreendido como uma consequência direta da estrutura quântica do vácuo e de sua sensibilidade às condições impostas pelo sistema físico. Mais do que um fenômeno isolado, o efeito Casimir emerge como uma ferramenta conceitual poderosa para investigar propriedades do vácuo em teorias quânticas de campos.

A formulação desenvolvida ao longo deste trabalho estabelece uma base sólida para extensões futuras do modelo considerado. Em particular, a introdução de uma dimensão extra compactificada associada a um campo éter permitirá investigar como modificações na estrutura do espaço-tempo afetam a energia do vácuo e o efeito Casimir. De forma complementar, a inclusão de um campo magnético externo exigirá a generalização da dinâmica do campo escalar por meio da substituição da derivada ordinária por uma derivada covariante, incorporando explicitamente o acoplamento com o potencial eletromagnético. Essas extensões abrirão caminho para o estudo de efeitos combinados de dimensões extras, campos externos e condições de contorno, contribuindo para uma compreensão mais ampla do papel do vácuo quântico em sistemas físicos não triviais.

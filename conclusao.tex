\chapter{Conclusões e Perpectivas}
Esta dissertação teve como objetivo estabelecer a ligação entre a formulação lagrangiana e as simetrias de uma teoria de campos, a construção do tensor energia--momento e da energia do vácuo, e a emergência do efeito Casimir quando o espectro de modos do campo é modificado pela presença de fronteiras.

O ponto de partida foi a dinâmica clássica de um campo escalar real descrita pela equação de Klein--Gordon \eqref{eq:KLeinGordonEq} e sua generalização na presença de uma fonte externa \eqref{eq:GenKleinGordonEq}. A partir desse formalismo, evidenciou-se que a informação essencial para a discussão do vácuo e de seus observáveis encontra-se na definição da ação e da densidade lagrangiana, Eqs.~\eqref{eq:acao} e \eqref{eq:lagphi}, e nas consequências das simetrias contínuas dessa ação.

Em particular, explorou-se a condição de invariância da ação sob transformações gerais \eqref{eq:tranformacoes}, com ênfase no caso em que a variação da lagrangiana se reduz a uma divergência total \eqref{eq:deltaL1}. A equivalência entre essa condição e a existência de uma corrente conservada foi explicitada ao comparar \eqref{eq:deltaL1} com a identidade obtida via Euler--Lagrange \eqref{eq:deltaL2}.

Como exemplo central, analisou-se a simetria de translação no espaço-tempo \eqref{eq:spacetimetranslation}, que induz a transformação infinitesimal do campo \eqref{eq:deltaPhi=epsilonPartialPhi} e implica a forma de divergência total \eqref{eq:partialLPartialK}. Nesse contexto, identificou-se o tensor energia--momento como a corrente de Nöther associada à translação, Eq.~\eqref{eq:stressTensor}, e sua conexão com o operador energia--momento è obtida pela integração espacial (ver, por exemplo, a relação final em \eqref{eq:HeP}). Essa etapa é conceitualmente decisiva, pois o efeito Casimir será extraídode valores esperados do tensor energia--momento e de sua dependência geométrica.

A transição para a teoria quântica foi realizada via quantização canônica e expansão em modos. Para contornos estáticos, introduziram-se soluções de frequência positiva e negativa \eqref{eq:campos} e o correspondente problema espectral espacial \eqref{eq:spatialEq}, com produto escalar \eqref{eq:scalarProduct} e condição de normalização \eqref{eq:normalization}. O operador de campo \eqref{eq:campogeral} foi construído em termos de operadores de criação e aniquilação que satisfazem as relações de comutação \eqref{eq:commutrel}, com definição do vácuo \eqref{eq:aKet0=0}. 

Nesse formalismo, a energia de vácuo surge como valor esperado da componente $00$ do tensor energia--momento \eqref{eq:tensorMeanvalue}. Substituindo-se a solução geral \eqref{eq:campogeral} em \eqref{eq:tensorMeanvalue}, obtém-se a expressão local \eqref{eq:<T>}, cuja integração no volume leva à soma sobre modos para a energia de ponto zero \eqref{eq:modeSum}. Em contrapartida, na ausência de fronteiras, a solução plana \eqref{eq:phiLivre} conduz à densidade de energia de vácuo de Minkowski \eqref{eq:freeSol}. Embora cada energia de vácuo seja divergente, as diferenças entre configurações com e sem restrições podem ser finitas e mensuráveis.

No Capítulo~2, o formalismo acima foi aplicado à geometria de duas placas paralelas com condições de contorno de Dirichlet. O espectro do campo é explicitamente modificado pela discretização do número de onda na direção normal às placas, o que se reflete na densidade de energia de vácuo entre as placas, Eq.~\eqref{eq:comPlacas}. A densidade correspondente do vácuo de Minkowski, escrita como integral contínua, è dada por \eqref{eq:semPlacas}. A desindade renormalizada \eqref{eq:denEnPlacas} foi obtida a partir da diferença de energia nas configurações com e sem placas usando Abel--Plana \eqref{eq:abelPlana}.

Ao descartar o termo oscilatório de \eqref{eq:denEnPlacas}, que não contribui para a energia total, obteve-se a energia de Casimir por unidade de àrea no limite não-massivo, Eq.~\eqref{eq:EnergiaEntrePlacas}. Finalmente, a pressão de Casimir foi extraída como derivada da energia em relação à separação entre as placas, culminando no resultado padrão \eqref{eq:pressaoEntreAsPlacas}. Esses resultados explicitam que o efeito Casimir è, essencialmente, um efeito espectral: as fronteiras alteram o conjunto de frequências \eqref{eq:campos} que contribuem na soma \eqref{eq:modeSum}, e a parte finita dessa reorganização espectral se manifesta como uma força (ou pressão) dependente de $a$.

\section{Perspectivas e trabalhos futuros}
\label{sec:perspectivas}

A estrutura desenvolvida ao longo do texto sugere extensões naturais, mantendo o mesmo núcleo metodológico: especificar a dinâmica (por exemplo, via \eqref{eq:KLeinGordonEq} ou \eqref{eq:GenKleinGordonEq}), fixar o observável de energia a partir de $T_{00}$ (como em \eqref{eq:tensorMeanvalue}--\eqref{eq:modeSum}) e caracterizar como condições geométricas e de contorno alteram o espectro (via \eqref{eq:spatialEq} e \eqref{eq:campos}), produzindo uma contribuição física finita após o procedimento de comparação/regularização (como ilustrado por \eqref{eq:comPlacas}--\eqref{eq:abelPlana}).

Uma primeira direção è estudar compactificações (por exemplo, dimensões extras) e campos adicionais que modifiquem as frequências permitidas, o que se traduz, na prática, em uma alteração do problema espectral \eqref{eq:spatialEq} e, portanto, da soma de modos \eqref{eq:modeSum}. Uma segunda direção è a inclusão de campos externos, que tipicamente alteram a forma efetiva da equação de movimento (e seu espectro), afetando as expressões locais de energia \eqref{eq:<T>} e os resultados integrados análogos a \eqref{eq:EnergiaEntrePlacas} e \eqref{eq:pressaoEntreAsPlacas}. Por fim, geometrias mais gerais e condições de contorno alternativas (Neumann, Robin ou mistas) podem ser tratadas dentro do mesmo arcabouço, modificando-se o conjunto de autofunções e frequências em \eqref{eq:campos}--\eqref{eq:spatialEq} e repetindo-se o procedimento de regularização do tipo \eqref{eq:abelPlana}. 

Em síntese, os resultados desta dissertação mostram que o efeito Casimir fornece um teste conceitual e técnico privilegiado para a teoria quântica de campos: a partir de uma definição bem controlada de energia e vácuo (Eqs.~\eqref{eq:tensorMeanvalue}--\eqref{eq:modeSum}), torna-se possível compreender como informações geométricas e de contorno entram no espectro do campo (Eqs.~\eqref{eq:campos} e \eqref{eq:spatialEq}) e como essa dependência se traduz em grandezas observáveis como a energia \eqref{eq:EnergiaEntrePlacas} e a pressão \eqref{eq:pressaoEntreAsPlacas}.

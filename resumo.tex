\newpage
\vspace*{5em}

\begin{center}
{\Huge {\bf Resumo}}
\end{center}

\vskip 4em

O efeito Casimir constitui uma das manifestações mais diretas da natureza quântica do vácuo, emergindo da quantização de campos na presença de condições de contorno. Neste trabalho, investigamos o efeito Casimir no contexto da teoria quântica de campos, com ênfase em campos escalares livres submetidos a geometrias confinadas. Inicialmente, apresentamos uma revisão dos fundamentos da teoria quântica de campos relativística, incluindo a formulação lagrangiana, a análise de simetrias contínuas e o teorema de Noether, estabelecendo a base conceitual necessária para a definição rigorosa da energia do sistema.

Em seguida, procedemos à quantização canônica do campo escalar e à interpretação da energia de ponto zero como uma soma sobre os modos normais do campo. A imposição de condições de contorno discretiza o espectro desses modos, dando origem a uma energia de vácuo dependente da geometria do sistema. Como aplicação explícita do formalismo desenvolvido, analisamos o efeito Casimir para um campo escalar confinado entre placas paralelas, obtendo a energia de Casimir e a pressão associada.

Os resultados apresentados evidenciam a sensibilidade da energia de vácuo às condições de contorno e reforçam o papel do efeito Casimir como uma ferramenta conceitual para o estudo da estrutura do vácuo quântico. O formalismo desenvolvido neste trabalho estabelece ainda uma base sólida para investigações futuras envolvendo extensões do modelo considerado, tais como modificações na topologia do espaço-tempo e a inclusão de campos externos.

\vspace*{2em}

\textbf{Palavras-chave}: Efeito Casimir; energia de vácuo; teoria quântica de campos; campos escalares; condições de contorno.
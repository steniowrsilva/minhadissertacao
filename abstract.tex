\vspace*{5em}

\begin{center}
{\Huge {\bf Abstract}}
\end{center}
\vskip 4em

The Casimir effect represents one of the most direct manifestations of the quantum nature of the vacuum, arising from the quantization of fields in the presence of boundary conditions. In this work, we investigate the Casimir effect within the framework of quantum field theory, with emphasis on free scalar fields subjected to confined geometries. We begin with a review of the fundamental aspects of relativistic quantum field theory, including the Lagrangian formulation, the analysis of continuous symmetries, and Noether’s theorem, thereby establishing the conceptual basis required for a rigorous definition of the system’s energy.

Subsequently, we perform the canonical quantization of the scalar field and interpret the zero-point energy as a sum over the normal modes of the field. The imposition of boundary conditions leads to a discretization of the mode spectrum, giving rise to a vacuum energy that depends on the geometry of the system. As an explicit application of the developed formalism, we analyze the Casimir effect for a scalar field confined between parallel plates, obtaining the Casimir energy and the associated pressure.

The results highlight the sensitivity of the vacuum energy to boundary conditions and reinforce the role of the Casimir effect as a conceptual tool for probing the structure of the quantum vacuum. Moreover, the formalism presented in this work provides a solid foundation for future investigations involving extensions of the model, such as modifications of spacetime topology and the inclusion of external fields.

\vspace*{2em}


\textbf{Keywords}: Casimir effect; vacuum energy; quantum field theory; scalar fields; boundary conditions.
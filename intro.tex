\chapter*{Introdução}
\addcontentsline{toc}{chapter}{Introdução}
O efeito Casimir constitui uma das manifestações mais claras da natureza física do vácuo quântico, emergindo como consequência direta da quantização de campos na presença de condições de contorno. Desde sua formulação original $\cite{Casimir1948}$, esse fenômeno tem desempenhado papel central tanto em investigações conceituais da teoria quântica de campos quanto em aplicações contemporâneas que envolvem sistemas mesoscópicos $\cite{Volovik2001}$, nanotecnologia $\cite{Passante2025}$ e modelos de dimensões extras $\cite{Decca2003}$. Em sua essência, o efeito Casimir explicita que o vácuo não é um estado vazio no sentido clássico: mesmo na ausência de partículas reais, permanecem flutuações quânticas associadas aos graus de liberdade do campo, e a energia de ponto zero passa a depender das propriedades geométricas, topológicas e das restrições impostas ao sistema.

Do ponto de vista físico, o cerne do efeito Casimir pode ser entendido como uma competição entre espectros de modos: ao se introduzir fronteiras (placas, cavidades, cascas esféricas, guias de onda, etc.) ou ao se compactificar dimensões, altera-se o conjunto de soluções admissíveis para a equação de movimento do campo $\cite{Asorey2006}$. Em uma teoria quântica, essa alteração não é “gratuita”: como cada modo carrega uma contribuição de energia de ponto zero, tipicamente $\tfrac{1}{2}\hbar\omega$, a reorganização do espectro leva a uma variação mensurável da energia do vácuo $\cite{Bordag2009}$. A força (ou pressão) de Casimir surge, então, como resposta do sistema à mudança de energia quando um parâmetro geométrico (como a separação entre placas) é modificado. Essa perspectiva é particularmente útil porque torna transparente a origem do fenômeno: não se trata de uma força fundamental adicional, mas de um efeito coletivo associado ao estado fundamental do campo sob restrições.

Em termos mais técnicos, a energia de ponto zero em teoria quântica de campos é, em geral, divergente. A soma $\sum_J \tfrac{1}{2}\hbar\omega_J$ sobre todos os modos cresce sem limite quando se inclui o contínuo de altas frequências (ultravioleta) $\cite{PeskinSchroeder1995}$. O que se observa fisicamente, entretanto, não é a energia absoluta do vácuo, mas diferenças de energia entre configurações distintas, por exemplo, “com placas” versus “sem placas”, ou uma dada geometria versus um espaço de referência. Essas diferenças podem ser finitas após um procedimento apropriado de regularização e renormalização $\cite{Srednicki2007}$, refletindo o fato de que apenas variações controladas da energia do vácuo são acessíveis experimentalmente. Essa característica torna o efeito Casimir um laboratório conceitual privilegiado: ele obriga a tratar com cuidado a noção de energia do vácuo, a interpretação física de quantidades divergentes e o papel das condições impostas pelo aparato experimental.

A relevância do efeito Casimir transcende o interesse puramente formal. Em escalas micro e nanométricas, forças de Casimir (e efeitos correlatos, como forças de van der Waals em regimes apropriados) podem tornar-se comparáveis a forças eletrostáticas e elásticas típicas de dispositivos $\cite{Banishev2010}$. Isso impacta diretamente o projeto e a confiabilidade de sistemas microeletromecânicos (MEMS/NEMS), onde a atração induzida pelo vácuo pode contribuir para instabilidades mecânicas, aderência (“stiction”) e limitação de desempenho $\cite{Serry1998}$. Em outra direção, o efeito também aparece em contextos de física matemática e física de altas energias: modelos de “bag” e confinamento efetivo $\cite{Solodukhin2002}$, teoria de campos em espaços com topologia não trivial $\cite{Bytsenko2001}$, presença de defeitos e fronteiras $\cite{Stefanazzi2024}$, bem como cenários com dimensões extras compactificadas, nos quais a quantização dos modos ao longo de uma dimensão compacta produz correções de energia interpretáveis como termos de Casimir \cite{DalvitLamoreaux2021}. Assim, o estudo sistemático do efeito Casimir fornece uma ponte entre aspectos fundamentais do vácuo quântico e consequências em sistemas físicos concretos.

Nesta dissertação, investigamos o efeito Casimir no contexto da teoria quântica de campos, com ênfase inicial em campos escalares submetidos a condições de contorno simples. O propósito dessa escolha é duplo. Primeiro, campos escalares constituem o cenário mais direto para explicitar a estrutura matemática do problema: as equações de movimento, a expansão modal e a definição do vácuo são formuladas de modo particularmente transparente. Segundo, as técnicas desenvolvidas no caso escalar  (construção do tensor energia-momento, quantização canônica, identificação e manipulação da energia de ponto zero, tratamento de somas de modos e subtrações de referência) servem como base para generalizações posteriores, incluindo geometrias mais complexas, condições de contorno mais gerais e acoplamentos externos.

O objetivo principal do presente trabalho é, portanto, estabelecer uma base conceitual e técnica sólida que permita, em etapas posteriores, generalizar o efeito Casimir para cenários fisicamente mais ricos, envolvendo modificações na estrutura do espaço-tempo e a presença de campos externos. Em termos operacionais, isso significa construir com rigor (i) a noção de energia do sistema a partir do formalismo lagrangiano e das simetrias, (ii) a quantização do campo e a identificação do conteúdo físico do vácuo, e (iii) um procedimento de cálculo que torne finitas as grandezas relevantes (energia de Casimir e pressão associada) em geometrias confinadas.

Para organizar essa construção, adotamos uma estratégia progressiva. Partimos de fundamentos da teoria clássica de campos e avançamos até o formalismo quântico necessário para interpretar e calcular a energia do vácuo. Nesse caminho, o teorema de Noether desempenha papel central: ao relacionar simetrias contínuas da ação a correntes conservadas, ele fornece o arcabouço conceitual para introduzir o tensor energia-momento e, consequentemente, uma definição controlada de energia em teoria de campos.

Em seguida, procedemos à quantização canônica do campo escalar, enfatizando a interpretação da energia do vácuo como soma sobre modos normais. A presença de fronteiras (ou, mais genericamente, a imposição de restrições de contorno) converte parte do espectro contínuo em espectro discreto e introduz dependência geométrica explícita nas frequências $\omega_J$. Essa discretização é justamente o mecanismo pelo qual o vácuo “responde” ao confinamento. Entretanto, como já mencionado, a soma direta sobre modos é divergente e exige ferramentas matemáticas de regularização. Assim, além de estabelecer o formalismo quântico, é necessário explicar com clareza por que e como se extrai a contribuição física finita: tipicamente, por comparação com uma configuração de referência (por exemplo, o espaço de Minkowski sem fronteiras) e pela aplicação de identidades e técnicas analíticas que reorganizam somas e integrais de modo controlado.

No escopo deste trabalho, escolhemos como aplicação explícita (e também como caso-teste) a configuração de placas paralelas, por ser a geometria prototípica onde o efeito Casimir se manifesta de maneira mais direta e onde a conexão entre espectro discreto e força resultante pode ser exposta com nitidez. O confinamento em uma direção espacial impõe condições de contorno que selecionam um conjunto particular de modos admissíveis, modificando o espectro ao longo do eixo perpendicular às placas. A energia de Casimir por unidade de área é obtida a partir do valor esperado do tensor energia-momento no vácuo, subtraindo-se a contribuição de referência e aplicando-se procedimentos analíticos que tornam a expressão finita; a pressão associada decorre então da derivada da energia em relação à separação entre as placas.

Além do cálculo em si, é importante destacar o lugar do efeito Casimir no panorama mais amplo da física teórica. A energia do vácuo aparece em discussões sobre estabilidade, estrutura do estado fundamental e respostas do sistema a perturbações geométricas. Em cenários com topologia não trivial, a energia de vácuo pode induzir termos efetivos que influenciam a dinâmica em baixas energias. Em modelos com dimensões extras compactificadas, as condições periódicas (ou quase-periódicas) ao longo de direções compactas geram espectros discretos análogos aos obtidos com placas, e a energia resultante pode ter implicações em mecanismos de estabilização de raios de compactificação e em correções efetivas observáveis em teorias de campo e gravitação. Dessa forma, o efeito Casimir funciona como um “sensor” teórico: ele torna visíveis, por meio de uma quantidade energética, informações sobre a estrutura geométrica e topológica do espaço-tempo e sobre as restrições físicas impostas aos campos.

Com base nessas motivações, a dissertação tem uma sequência consistente com elas. O Capítulo 1 é dedicado a uma revisão sistemática dos fundamentos necessários para o estudo do efeito Casimir. Iniciamos com a formulação lagrangiana de campos clássicos e a análise das simetrias contínuas da ação, destacando o papel do teorema de Noether na construção do tensor energia-momento e na definição rigorosa da energia do sistema. Em seguida, procedemos à quantização canônica do campo escalar e à interpretação da energia do vácuo como soma sobre modos normais, estabelecendo o arcabouço formal que sustenta o cálculo da energia de Casimir.

No Capítulo 2, o efeito Casimir é analisado explicitamente no caso de um campo escalar confinado entre placas paralelas, onde as condições de contorno discretizam o espectro de modos do campo. A partir desse formalismo, obtêm-se a energia do vácuo e a pressão de Casimir associada, evidenciando a dependência do efeito em relação às propriedades geométricas do sistema. O objetivo aqui é apresentar um cálculo completo e consistente, no qual cada etapa (da escolha de modos à obtenção de uma grandeza finita) seja motivada e interpretada fisicamente.

Por fim, o Capítulo 3 sintetiza as conclusões e discute perspectivas. A estrutura desenvolvida nesta dissertação foi concebida de modo a permitir extensões naturais do modelo básico. Em trabalhos futuros, pretende-se investigar o efeito Casimir na presença de um campo éter associado a uma dimensão extra compactificada, com condições de quase-periodicidade ao longo da quinta dimensão, conforme sugerido em modelos de teorias efetivas com quebra suave de simetrias de Lorentz. Além disso, a inclusão de um campo magnético externo exigirá a modificação da dinâmica do campo escalar por meio da introdução de uma derivada covariante, incorporando explicitamente o acoplamento mínimo com o quadrivetor potencial eletromagnético. Essas generalizações permitirão explorar como a estrutura do vácuo quântico é afetada por dimensões extras e campos externos, ampliando o alcance físico dos resultados obtidos.

